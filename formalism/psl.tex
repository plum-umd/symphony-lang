\documentclass[10pt]{article}

\usepackage{times}
\usepackage{fullpage}
\usepackage{amsmath}
\usepackage{amssymb}
\usepackage{mathpartir}
\usepackage{color}

%%%%% DEFS %%%%%%

\newcommand{\rulelab}[1]{{\small \textsc{#1}}}
\newcommand{\kw}[1]{\ensuremath{\mathtt{#1}}}

% Types
\newcommand{\tnat}{\ensuremath{\mathtt{nat}}}
\newcommand{\tfun}[3]{\ensuremath{{#1} ~^{#3}\!\rightarrow {#2}}}
\newcommand{\tprod}[2]{\ensuremath{{#1} \times {#2}}}
%\newcommand{\tsum}[4]{\ensuremath{{#1} +^{#3}_{#4} {#2}}}
\newcommand{\tsum}[3]{\ensuremath{{#1} +^{#3} {#2}}}
\newcommand{\trec}[2]{\ensuremath{\mu {#1}.{#2}}}
% \newcommand{\ssec}{\ensuremath{\mathtt{\cdot}}}
% \newcommand{\isec}{\ensuremath{\mathtt{pmap}}}
% \newcommand{\sshare}[1]{\ensuremath{\mathtt{shr}~{#1}}}
% \newcommand{\sectyp}[3]{\ensuremath{{#1} \{~{#2}:{#3}~\}}}
  
% Terms
\newcommand{\ebinop}[2]{\ensuremath{{#1}~\oplus~{#2}}}
\newcommand{\elet}[3]{\ensuremath{\kw{let}~#1\, =\, #2~\kw{in}\;{#3}}}
\newcommand{\epar}[2]{\ensuremath{\kw{par}~{#1}~{#2}}}
\newcommand{\esec}[2]{\ensuremath{\kw{sec}~{#1}~{#2}}}
\newcommand{\ereveal}[4]{\ensuremath{\kw{reveal}^{#1}_{#4}~{#2}~{#3}}}
\newcommand{\econd}[3]{\ensuremath{\kw{match}~{#1}~\kw{with}~x.{#2} \diamond {#3}}}
%\newcommand{\emux}[3]{\ensuremath{\kw{mux}~{#1}~\kw{?}~x.{#2}~\kw{:}~x.{#3}}}
\newcommand{\emux}[3]{\ensuremath{\kw{mux}~{#1}~\kw{?}~{#2}~\kw{:}~{#3}}}
\newcommand{\eshare}[4]{\ensuremath{\kw{share}^{#2}_{#1}~{#3}~{#4}}}
%\newcommand{\esharesum}[4]{\ensuremath{\kw{sharesum}^{#2}_{#1}~{#3}~{#4}}}
\newcommand{\einj}[2]{\ensuremath{\kw{inj}_{#1}~{#2}}}
\newcommand{\eread}{\ensuremath{\kw{read}}}
\newcommand{\ewrite}[1]{\ensuremath{\kw{write}~{#1}}}
\newcommand{\epair}[2]{\ensuremath{\langle {#1}, {#2} \rangle}}
\newcommand{\eproj}[2]{\ensuremath{\kw{\#}}_{#1}~{#2}}
\newcommand{\elam}[2]{\ensuremath{\lambda {#1}.{#2}}}
\newcommand{\eapp}[2]{\ensuremath{{#1}~{#2}}}
\newcommand{\efix}[3]{\ensuremath{\kw{fix}~{#1}.\elam{#2}{#3}}}
\newcommand{\efold}[2]{\ensuremath{\kw{fold}_{#1}~{#2}}}
\newcommand{\eunfold}[1]{\ensuremath{\kw{unfold}~{#1}}}
\newcommand{\vshare}[3]{\ensuremath{\{{#3}\}^{#1}_{#2}}}
\newcommand{\vloc}[2]{\ensuremath{{#1}\kw{@}{#2}}}
\newcommand{\vclos}[2]{\ensuremath{\mathbf{clos}~({#1},{#2})}}

% Judgments
\newcommand{\hastyp}[4]{\ensuremath{{#1} \vdash_{#2} {#3} : {#4}}}
%\newcommand{\canshare}[1]{\ensuremath{\text{isFlat } {#1}}}
\newcommand{\subtype}{\ensuremath{<:}}
\newcommand{\issub}[2]{{#1} \subtype {#2}}
\newcommand{\eval}[4]{\ensuremath{\config{#1}{#3} \longrightarrow_{#2} {#4}}}

% Aux
\newcommand{\env}{\ensuremath{\sigma}}
\newcommand{\config}[2]{\ensuremath{\langle{#1},{#2}\rangle}}
\newcommand{\locof}[2]{\ensuremath{\mathit{loc}_{#1}~{#2}}}
\newcommand{\getat}[2]{\ensuremath{\mathit{on}_{#1}~{#2}}}
% \newcommand{\ctxt}{\ensuremath{\mathcal{E}}}
% \newcommand{\pctxt}{\ensuremath{\mathcal{P}}}
  
%%%%%%%%%%

\newcommand{\mwh}[1]{\textcolor{red}{Mike: #1}}

\title{PANTHEON Source Language}
\author{PANTHEON Team}

\begin{document}

\maketitle

\section{TO DO}

Soon:
\begin{itemize}
\item David D says: ``This looks good to me, but the last $v = loc_{m
    \cap P} v_0$ premise on \rulelab{E-Par} may not be necessary. (It may be a lemma that is
  true about any $e \longrightarrow_m v$.)
\item Develop multi-party operational semantics.
\item Prove relationship between single- and multi-threaded semantics
\item Prove type soundness for single-threaded semantics
\end{itemize}

% Next: Look at \texttt{writeup/spdz.md} and think about language
% changes to support verification.

\noindent
Later:
\begin{itemize}
\item Is there any benefit to the $m \vdash m'$ in T-Abs? Not
  convinced. Reconsider function types --- see DavidH's note below.
\item No subtyping at present for recursive types. Should we add
  it?
\item Contemplate Muxable t psi, which denotes that there must be some
  way to combine two different values of type t according to a mux
  under protocol psi. From David D: \emph{Regarding muxable: I still see $A +^P_{\psi} B$ as a sum type where psi only describes the encryption of the "sum bit", and $A$ and $B$ should either be shared base types, tuples, or shared sums (arbitrarily nested). This is the muxable constraint, and I think what David H had in mind.}
\item Contemplate generalized ``reveal'' operation, which works on all
  datatypes by recursing into their contents and revealing
  those. E.g., you could reveal a list of shares all at once.
\item First-class principal sets, as data,
  \begin{itemize}
    \item Add to syntactic forms (e.g., $\ereveal{P}{Q}{e}{\psi}$ should be
    $\ereveal{e_q}{e_q}{e}{\psi}$ where $e_P$ and $e_q$ are principal
    sets).
  \item Generalize wire bundles.
  \end{itemize}
\item Function types preclude top-level functions? See note below.
\end{itemize}

\begin{verbatim}
By the way, I don't know if this will overcomplicate everything, but
I've been thinking about how to get back the notion of applying a
function with fewer parties than were there when the closure was
made. (Which if I understand correctly is not possible in the current
version since function sub-typing is invariant in the knownness of the
function arrow). 

It seems that you can view the annotation ^m (on both values and
contexts) as a sort of upper bound on the parties who know some
information (which you can subtype to include fewer parties). 
Analogously you could imagine adding an annotation _n to represent a
lower bound on the parties who know information (conceptually, the
parties whose presence is mandatory to use a value). 
Then you could give function arrows this subtype rule:

t1' <: t1
t2 <: t2'
m >= m'
n' >= n
-------------------
t1 ->^m_n t2    <:    t1' ->^m'_n' t2'

The application rule would require the upper and lower bound to match: 

gamma |-^m_n e : t1 ->^m_m t2
gamma |-^m_n e1 : t1
------------------
gamma |-^m_n (e e1) : t2

par blocks would constrain the lower bound: By including a par block
as a subexpression you indicating that the containing expression must
involve the par block parties: 

m >= m'
gamma |-^m'_n e : t
--------------
gamma |-^m_(n \cup m') par m' e : t
\end{verbatim}


\begin{figure}[h]
  \centering
  \[\begin{array}{rlcll}
      \text{Principal} & p, q \\
      \text{Principal set} & P, Q \\
    \text{Execution modes} & m  & ::=  & $P$ & \text{par mode with principal set $P$ (could be $\emptyset$)} \\
      \text{Protocol} & \psi & ::= & \cdot & \text{cleartext} \\
                       && \mid & \phi & \text{encryption format} \\
      \text{Types} & \tau & ::=  & \tnat^m_\psi & \text{a base type} \\
                       && \mid & \tprod{\tau}{\tau} & \text{pairs} \\
%                       && \mid & \tsum{\tau}{\tau}{m}{\psi} & \text{sums} \\
                       && \mid & \tsum{\tau}{\tau}{m} & \text{sums} \\
                       && \mid & \trec{\alpha}{\tau} & \text{(iso)recursive types} \\
                       && \mid & \alpha & \text{type variables (for recursive types)} \\
                       && \mid & \tfun{\tau}{\tau}{m} & \text{functions} \\
      \text{Protocol} & \psi & ::= & ... & \text{MPC type} \\
      \text{Expressions} & e & ::= & n & \text{naturals} \\
                       && \mid & x & \text{variables} \\
                       && \mid & \eread & \text{read from terminal}\\
                       && \mid & \ewrite{e} & \text{write to terminal}\\
                       && \mid & \epair{e_1}{e_2} & \text{pair construction}\\
                       && \mid & \eproj{i}{e} & \text{pair elem, }i \in \{1,2\}\\
                       && \mid & \eshare{\psi}{P}{Q}{e} & \text{$\psi$ shares of {\tnat} at $P$ to $Q$} \\
                       % && \mid & \esharesum{\psi}{P}{Q}{e} & \text{$\psi$ shares of sum $P$ to $Q$} \\
                       && \mid & \ereveal{P}{Q}{e}{\psi} & \text{distribute result in $\psi$ at $Q$ to $P$}\\
                       && \mid & \ebinop{e_1}{e_2}  & \text{binary operation} \\
                       && \mid & \emux{e}{e_1}{e_2}  & \text{MPC conditional} \\
                       && \mid & \einj{i}{e} & \text{sum elem, }i \in \{1,2\}\\
                       && \mid & \econd{e}{e_1}{e_2}  & \text{sum elimination} \\
                       && \mid & \efold{\trec{\alpha}{\tau}}{e} & \text{rectype intro}\\
                       && \mid & \eunfold{e} & \text{rectype elim}\\
                       && \mid & \epar{P}{e} & \text{parallel evaluation}\\
                       && \mid & \elam{x}{e}  & \text{abstraction} \\
                       && \mid & \eapp{e_1}{e_2}  & \text{application} \\
                       && \mid & \elet{x}{e_1}{e_2}  & \text{sequencing} \\
  \end{array}
  \]
  \caption{Syntax}
  \label{fig:syntax}
\end{figure}

\newpage

\section{Syntax}
  
The syntax of core PSL is in Figure~\ref{fig:syntax}. Some derived
constructs (syntactic sugar) are given in Section~\ref{sec:derived}

There are two key concepts in types. First is their \emph{location},
designated $m$. Natural numbers, sums, and functions are located at
particular places; only here can they be computed on. Second (and in
addition), natural numbers can be \emph{encrypted} as secret
shares (with a different share at each party among the locations
$m$). We annotate the sharing protocol (GMW, Yao, etc.) as $\psi$; the
``cleartext'' protocol is $\cdot$ (or simply elided, to reduce
clutter). Pairs and lists are not located in the same sense as the
other values. That is, all parties present in a computation when the
pair or list is created can see its structure. They may not, however,
be able to access its values, as these could be located at particular
locations.

\section{Semantics (Single Threaded)}

\begin{figure}
  \[\begin{array}{rlcll}
      \text{Store} & \sigma & \in & \mathbf{Var} \rightharpoonup \mathbf{Val}\\
      \text{Locatable value} & u & ::=  & n & \text{numbers} \\
                             && \mid & \einj{i}{v} & \text{sum elem, }i \in \{1,2\}\\
                             && \mid & \vclos{\env}{e}  & \text{closure} \\
                             && \mid & \vshare{Q}{\psi}{e} & \text{$\psi$-$Q$ share of evaluating $e$} \\
      \text{Value} & v  \in \mathbf{Val} & ::=  & u & \text{locatable value}\\
                       && \mid & \epair{v_1}{v_2} & \text{pair value}\\
                       && \mid & \efold{\trec{\alpha}{\tau}}{v} & \text{rectype value}\\
                       && \mid & \vloc{u}{P} & \text{located value}\\
      % \text{Eval Context} & \ctxt & ::= & \multicolumn{2}{l}{\bullet \mid\epair{\ctxt}{e} \mid
      %                        \epair{v}{\ctxt} \mid \eproj{i}{\ctxt}}  \\
      %          && \mid & \multicolumn{2}{l}{\eshare{\psi}{P}{Q}{\ctxt} \mid
      %                        \ereveal{P}{Q}{\ctxt}{\psi} \mid \ebinop{\ctxt}{e_2} \mid \ebinop{v}{\ctxt} }  \\
      %          && \mid & \multicolumn{2}{l}{\emux{\ctxt}{e_1}{e_2}
      %                    \mid \emux{v}{\ctxt}{e_2} \mid \emux{v}{v_1}{\ctxt}  }  \\
      %          && \mid & \multicolumn{2}{l}{\einj{i}{\ctxt} \mid \econd{\ctxt}{e_1}{e_2} }  \\
      %          && \mid & \multicolumn{2}{l}{\efold{\trec{\alpha}{\tau}}{\ctxt} \mid \eunfold{\ctxt}}  \\
      %          && \mid & \multicolumn{2}{l}{\eapp{\ctxt}{e} \mid
      %                    \eapp{v}{\ctxt} \mid \elet{x}{\ctxt}{e_2} }\\
    \end{array}
  \]

\[\begin{array}{l@{~~=~~}l}
    \locof{P}{n} & \vloc{n}{P} \\
    \locof{P}{\einj{i}{v}} & \vloc{(\locof{P}{v})}{P} \\
    \locof{P}{\vclos{\env}{\elam{x}{e}}} & \vloc{(\vclos{(\locof{P}{\env})}{\elam{x}{e}})}{P}\\
    \locof{P}{\vshare{Q}{\psi}{e}} & \vloc{(\vshare{Q}{\psi}{e})}{P} \\
    \locof{P}{\epair{v_1}{v_2}} & \epair{\locof{P}{v_1}}{\locof{P}{v_2}}\\
    \locof{P}{ \efold{\trec{\alpha}{\tau}}{v}} &  \efold{\trec{\alpha}{\tau}}{\locof{P}{v}}\\
    \locof{P}{\vloc{u}{Q}} & \vloc{(\locof{P}{u})}{R} \qquad\text{where }R = P \cap Q\\
    \locof{P}{\sigma} & \{ x \mapsto v' \mid \sigma(x) = v \land \locof{P}{v} = v' \}\\
    \multicolumn{2}{c}{}\\
    \getat{P}{(\vloc{u}{Q})} & u \qquad \text{where }Q \vdash P\\
    \getat{P}{v} & v \qquad \text{for all other $v$ syntactic forms}\\
  \end{array}
\]
\caption{Semantics auxiliaries}
\label{fig:auxsem}
\end{figure}

\begin{figure}
$$
\begin{array}{c}
    \inferrule*[lab=E-Nat]
    {
    }
    {
    \eval{\env}{m}{n}{n}
    }
    \qquad

    \inferrule*[lab=E-Var]
    {
    \env(x) = v
    }
    {
    \eval{\env}{m}{x}{v}
  }\qquad

  \inferrule*[lab=E-Read]
    {
  m\text{ is a singleton}
    }
    {
    \eval{\env}{m}{\eread}{n}
    }
    \qquad

      \inferrule*[lab=E-Write]
      {
      m\text{ is a singleton}\\\\
      \eval{\env}{m}{e}{v}\\
  \getat{m}{v} = n
    }
    {
    \eval{\env}{m}{\ewrite{e}}{n}
    }
    \\ \\

    \inferrule*[lab=E-Let]
    {
    \eval{\env}{m}{e_1}{v_1}\\\\
    \eval{\env[x\mapsto v_1]}{m}{e_2}{v_2}
    }
    {
    \eval{\env}{m}{\elet{x}{e_1}{e_2}}{v_2}
    }\qquad
  
    \inferrule*[lab=E-Pair]
    {
    \eval{\env}{m}{e_1}{v_1}\\\\
    \eval{\env}{m}{e_2}{v_2}
    }
    {
    \eval{\env}{m}{\epair{e_1}{e_2}}{\epair{v_1}{v_2}}
    }\qquad
    
    \inferrule*[lab=E-Access]
    {
    \eval{\env}{m}{e}{v}\\\\
    \getat{m}{v} = \epair{v_1}{v_2}
    }
    {
    \eval{\env}{m}{\eproj{i}{e}}{v_i}
    }\\\\

    \inferrule*[lab=E-inj]
    {
    \eval{\env}{m}{e}{v}
    }
    {
    \eval{\env}{m}{\einj{i}{e}}{\einj{i}{v}}
    } \qquad
    
    \inferrule*[lab=E-Match]
    {
    \eval{\env}{m}{e}{v_0}\\
    \getat{m}{v_0} = \einj{i}{v_i}\\\\
    \eval{\env[x\mapsto v_i]}{m}{e_i}{v}
    }
    {
      \eval{\env}{m}{\econd{e}{e_1}{e_2}}{v}
    }\\ \\

    \inferrule*[lab=E-Fold]
    {
    \eval{\env}{m}{e}{v}
    }
    {
    \eval{\env}{m}{\efold{\trec{\alpha}{\tau}}{e}}{\efold{\trec{\alpha}{\tau}}{v}}
    } \qquad

    \inferrule*[lab=E-Unfold]
    {
    \eval{\env}{m}{e}{v_0}\\\\
    \getat{m}{v_0} = \efold{\trec{\alpha}{\tau}}{v}
    }
    {
    \eval{\env}{m}{\eunfold{e}}{v}
    } \qquad

    \inferrule*[lab=E-Share]
    {
    m \vdash P\\ m \vdash Q\\\\
    P\text{ is a singleton}  \\\\
    \eval{\env}{m}{e}{v}\\
    \getat{P}{v} = n  
    }
    {
  \eval{\env}{m}{\eshare{\psi}{P}{Q}{e}}{\vloc{\vshare{Q}{\psi}{n}}{Q}}
    }\\ \\
    
    \inferrule*[lab=E-BinopNat]
    {
  \eval{\env}{m}{e_1}{v_1}\\
  \getat{m}{v_1} = n_1\\\\
    \eval{\env}{m}{e_2}{v_2}\\
  \getat{m}{v_2} = n_2\\\\
  n = n_1 \oplus n_2\\
    }
    {
    \eval{\env}{m}{\ebinop{e_1}{e_2}}{n}
    }\qquad

      
    \inferrule*[lab=E-BinopShare]
    {
  \eval{\env}{m}{e_1}{v_1}\\
  \getat{m}{v_1} = \vshare{m}{\psi}{e'_1} \\\\
    \eval{\env}{m}{e_2}{v_2}\\
  \getat{m}{v_2} = \vshare{m}{\psi}{e'_2} \\\\
  v = \vshare{m}{\psi}{\ebinop{e'_1}{e'_2}}
    }
    {
    \eval{\env}{m}{\ebinop{e_1}{e_2}}{v}
    }\\\\

    \inferrule*[lab=E-Mux]
    {
  \eval{\env}{m}{e}{v}\\
  \getat{m}{v} = \vshare{m}{\psi}{e'} \\\\
  \eval{\env}{m}{e_1}{v_1}\\
  \getat{m}{v_1} = \vshare{m}{\psi}{e'_1} \\\\
    \eval{\env}{m}{e_2}{v_2}\\
  \getat{m}{v_2} = \vshare{m}{\psi}{e'_2} \\\\
  v = \vshare{m}{\psi}{\emux{e'}{e_1'}{e_2'}}
    }
    {
    \eval{\env}{m}{\emux{e}{e_1}{e_2}}{v}
    }\qquad

    \inferrule*[lab=E-Reveal]
  {
    m \vdash P\\ m \vdash Q\\\\
    \eval{\env}{m}{e}{v_0}\\
    \getat{P}{v_0} = \vshare{P}{\psi}{e_0}\\\\
    \eval{\env}{m}{e_0}{v}\\
    n = \getat{P}{v} 
    } 
    {
    \eval{\env}{m}{\ereveal{P}{Q}{e}{\psi}}{\vloc{n}{Q}}
    }\\\\
    
   \inferrule*[lab=E-Abs]
    {
    }
    {
    \eval{\env}{m}{\elam{x}{e}}{\vclos{\env}{\elam{x}{e}}}
    }\qquad
   
    \inferrule*[lab=E-Fix]
    {
    }
    {
    \eval{\env}{m}{\efix{f}{x}{e}}{\vclos{\env}{\efix{f}{x}{e}}}
    }\\\\

    \inferrule*[lab=E-App]
    {
  \eval{\env}{m}{e}{v'}\\
    \getat{m}{v'} = \vclos{\env'}{\elam{x}{e'}}\\\\
  \eval{\env}{m}{e_1}{v_1}\\
  \eval{\env'[x \mapsto v_1]}{m}{e'}{v}\\
    }
    {
    \eval{\env}{m}{\eapp{e}{e_1}}{v}
    }\qquad

  
    \inferrule*[lab=E-FixApp]
    {
  \eval{\env}{m}{e}{v'}\\
    \getat{m}{v'} = \vclos{\env'}{\efix{f}{x}{e'}}\\\\
 \eval{\env}{m}{e_1}{v_1}\\
 \eval{\env'[f \mapsto v'][x \mapsto v_1]}{m}{e'}{v}\\
    }
    {
    \eval{\env}{m}{\eapp{e}{e_1}}{v}
    }\\ \\

    \inferrule*[lab=E-Par]
    {
  Q = m \cap P\\\\
  Q = \emptyset \Rightarrow v = \locof{Q}{0}\\\\
    Q \not= \emptyset \Rightarrow
    \env' = \locof{P}{\env} \land
    \eval{\env'}{{Q}}{e}{v}
    }
    {
    \eval{\env}{m}{\epar{P}{e}}{v}
    } \qquad
\end{array}
    $$
\caption{Operational Semantics}
\label{fig:sem}
\end{figure}

We define a big-step operational semantics. The judgment
$\eval{\env}{m}{e}{v}$ states that program
$e$ evaluates under mode $m$ and store $\env$ to a value
$v$. The rules are given in Figure~\ref{fig:sem}.  This is the
``single threaded semantics'' in that we don't have independently
executing parties; rather, we simulate the group of principals
$m$ executing in lockstep.

To be clear about which values reside on which parties' hosts, we have
a special value form to indicates this. Values are defined in
Figure~\ref{fig:auxsem}. The form $\vloc{u}{P}$ indicates that $u$ is
only visible to principals $p \in P$. Since not all values need to
be explicitly located (if they are in scope, they implicitly
are accessible), we distinguish \emph{locatable} values $u$.
Values for numbers, sums, and closures are standard (closures
have an explicit environment---we don't use substitution), as are
(locatable) values for pairs and recursive types. The value
$\vshare{Q}{\psi}{e}$ represents a circuit for a secure computation
under scheme $\psi$ to be performed by principals in $q \in Q$. Notice
that it contains an expression $e$, not a value $v$; this represents a
``suspended'' computation. 

In the same figure are two functions over values: $\locof{P}{v}$ and
$\getat{v}{P}$. The first is a transformation of $v$ that
``relocates'' it to $P$. For locatable, but not located, values $u$,
we annotate them with $P$. For compound forms we recurse inside them
(sums, pairs, closure environments). Locating an environment locates
its mapped-to values pointwise.  For located values $\vloc{u}{Q}$ we
relocate $u$'s contents and update its location to be $P$ intersected
with $Q$. Note that this intersection could be the $\emptyset$ which
indicates an inaccessible value.

The function $\getat{v}{P}$ attempts to strip off the outermost
location designator so the value $v$ can be computed on. For all
values other than $\vloc{u}{Q}$ this is a no-op. For these, we confirm
that the location $Q$ is compatible with the accessing
environment. This is never true if the value's location $Q$ is
$\emptyset$; it will be true if $Q$ contains all principals in the
requested mode $P$.

Now we turn to the rules. Many of them are essentially standard, in
particular \rulelab{E-Nat}, \rulelab{E-Pair}, \rulelab{E-Inj},
\rulelab{E-Var}, \rulelab{E-Let}, \rulelab{E-Abs}, and \rulelab{E-Fix}. Many others are
\emph{almost} standard, including \rulelab{E-Write}, \rulelab{E-Access},
\rulelab{E-Match}, \rulelab{E-Unfold}, \rulelab{E-BinopNat}, 
\rulelab{E-App}, and \rulelab{E-FixApp}. These differ from their
standard counterparts in that 
that refer to $\getat{m}{v}$ in the premise---as elimination forms,
they have to have to strip off any location information to use the
value. This operation will fail if the value is not available to
(located at) all principals in the current mode $m$; the type system
aims to rule out this sort of problem. \rulelab{E-Read} and
\rulelab{E-Write} are unsurprising; notably, they only work in a
singleton context---we need to know 
which principal is reading from/writing to their terminal. 

Skipping ahead, consider \rulelab{E-Par}, which evaluates
$\epar{P}{e}$. It evaluates $e$ in mode $Q = m \cap P$ to produce
$v_0$; i.e., only those principals in $P$ \emph{also} present $m$ will
run $e$. If $Q$ is empty, then no evaluation takes place, and $0$
(located at $Q = \emptyset$) is
returned. (The alternative of running $e$ with $\emptyset$ as the mode
is not equivalent: Skipping evaluation means avoiding the possibility
of non-termination or getting stuck on other rules due to inaccessible
values.) Otherwise, evaluation takes place in environment $\sigma'$,
which is the current environment located at $P$, which is constructed
by intersecting mapped-to values' locations with $P$. This rule
maintains and relies on an important invariant: \emph{the current
  environment $\sigma$ is always compatible (located at) the current
  mode $m$.} As such, since $\sigma$ is compatible with $m$ already,
it just needs to also be located at $P$, and the result will be
compatible with $Q$. This invariant is also needed, for example, for
\rulelab{E-Abs} to make sense, since it captures the current
environment to store in the produced closure. Note that the result
$v$ need not be explicitly located at $Q$ before it can be returned;
this is another invariant---any value returned from executing in mode
$m$ is compatible with $m$ (i.e., it will not be located at any
principal $p \not\in m$).

Let's consider the remaining rules, which focus on multiparty
computation. \rulelab{E-Share} models principal $p \in P$
``encrypting'' an integer, sharing it with principals $Q$. Expression
$e$ is evaluated in the current mode $m$, but the direction is that
just $P$ is doing the sharing; hence we check that $P$ is present in
$m$, and $v$ is located
on $P$. We then encapsulate the extracted number $n$ in a share value,
split between principals at $Q$, which must be present in $m$
(ensuring our invariant on the location of final values).

\rulelab{E-BinopShare} permits multiparty computation on shares. It
makes sure both arguments are available to exactly the executing hosts
and the encapsulates the ``suspended'' computation in a share
itself. \rulelab{E-Mux} produces a multiplexor on shares. It evaluates
its arguments $e$, $e_1$, and $e_2$ to shares, and then constructs a
circuit involving all three.

\rulelab{E-Reveal} eliminates shares by ``forcing'' the suspended
computation in the given share. We require that the principals $P$ who
each have a share are present in $m$, and likewise the principals in
$Q$ to whom the final result is sent.

\section{Semantics (Multi-threaded)}

\mwh{TODO. Thoughts follow.}

Each party $p$ has its own store $\env_p$ and program $e_p$, rather
than there being a global program. Evaluating these is equivalent to
running with $m = \{ p \}$, i.e., for a single party, roughly
speaking.

A key change in the design of ST mode would be to make it that if you
start the same program, meant to be run by $p$ and $q$ together, with
just $p$ or just $q$, it will do so correctly. This will happen as
long as the $\kw{par}$ mode rule is correct: It will just subset the
annotation $P$ on the $\kw{par}$ with the current mode---a
singleton---and then that single party will continue on and do the
right thing. 

Computations that require coordination will be done in a mode that
contains the coordinating parties. The single-mode parties must
synchronize before carrying out the computation. They will wait for
each party to reach the same spot, carry out the computation, and
continue on with whatever their local result should be. The type
system should ensure things line up.

More details:
\begin{itemize}
\item $\eshare{P}{\psi}{Q}{e}$ requires coordination among $P \cup Q$,
  where $P$ is a singleton. Somehow, the result of this computation
  will be $\vshare{Q}{\psi}{n}$ and given to each party $q \in Q$,
whereas $P$ gets $0$ (if he is not in $Q$).
\item $\ebinop{e_1}{e_2}$ is just as in the ST mode---each party in
  the share will have the full copy of it. Same with $\kw{mux}$.
\item $\ereveal{P}{\psi}{Q}{e}$ requires coordination among $P \cup
  Q$, where $e$ is a $\vshare{Q}{\psi}{e'}$---i.e., the $Q$ parties
match up. Each of the parties $q \in Q$ should reach this redex
\emph{and have the same $e'$}. The type system should ensure they
agree on this, the circuit that's been created. Moreover, $e'$ should
only consist of numbers, binops, and muxes. This means you can just
\emph{run it without an environment, in an empty mode}. We let every
$q \in Q$ do that, and thus they will agree on the result $n$. Parties
$p \in P$ then receive this result, while any other parties $q \not\in
P$ receive $0$.
\end{itemize}

I'm a little unsure about the communication between parties. How
should this be encoded? Is there a generic way to do it? This will
happen at $\kw{share}$ and $\kw{reveal}$. I'm also unsure that the
``just intersect the principals at par mode at will work'' jives with
the handling of these constructs. Need to mentally work through.

To prove that this semantics matches the ST one, I think we need some
kind of ``slice'' of a program, for each of the involved hosts. At
present, I think this slice can be static---you take the program,
create a slice of it, and then run each slice. When we have dependent
types, that may not be true anymore.

\section{Examples}

\mwh{More}

\begin{verbatim}
(fun x -> { par : C } x) 
  ({ par : A, B, C } reveal{ C } (share{ yao : A, B } 0))
\end{verbatim}
This is allowed by the type system. The $\kw{reveal}$ produces a value
at $C$, which may be ``passed through'' the par block at \verb+{A, B,C}+. 
This is then passed in and used in the function.

\begin{verbatim}
par(A,B) 
  let x = par(A) 0 in
  let y = par(B) 1 in
  par(A) 
    let z = (x,y) in #1 z
\end{verbatim}
Should work.

\begin{verbatim}
(fun x -> { par : B } ((), x)) ({ par : A } 0)
\end{verbatim}
The above is allowed. And equivalently, so is this (which in a
previous incarnation would have been rejected):
\begin{verbatim}
{ par : B } ((), ({ par : A } 0))
\end{verbatim}

\section{Typing}

\begin{figure}
\[\begin{array}{c}

    \inferrule*[lab=T-Nat]
    {
    }
    {
    \hastyp{\Gamma}{m}{n}{\tnat^{m}}
    } \qquad

    \inferrule*[lab=T-Var]
    {
    x\!:\!\tau \in \Gamma
    }
    {
    \hastyp{\Gamma}{m}{x}{\tau}
    }\qquad

    \inferrule*[lab=T-Read]
    {
    m\text{ is a singleton}
    }
    {
    \hastyp{\Gamma}{m}{\eread}{\tnat^{m}}
    }
    \qquad

    \inferrule*[lab=T-Write]
    {
    m\text{ is a singleton}\\\\
    \hastyp{\Gamma}{m}{e}{\tnat^{m}}
    }
    {
    \hastyp{\Gamma}{m}{\ewrite{e}}{\tnat^{m}}
    }

    \\ \\
    \inferrule*[lab=T-Let]
    {
    \hastyp{\Gamma}{m}{e_1}{\tau_1}\\\\
    \hastyp{\Gamma, x\!:\!\tau_1}{m}{e_2}{\tau}
    }
    {
    \hastyp{\Gamma}{m}{\elet{x}{e_1}{e_2}}{\tau}
    }\qquad

    \inferrule*[lab=T-Pair]
    {
    \hastyp{\Gamma}{m}{e_1}{\tau_1}\\\\
    \hastyp{\Gamma}{m}{e_2}{\tau_2}
    }
    {
    \hastyp{\Gamma}{m}{\epair{e_1}{e_2}}{\tau_1 \times \tau_2}
    }\qquad
    
    \inferrule*[lab=T-Access]
    {
    \hastyp{\Gamma}{m}{e}{\tau_1 \times \tau_2}\\
    }
    {
    \hastyp{\Gamma}{m}{\eproj{i}{e}}{\tau_i}
    }\\\\

    \inferrule*[lab=T-inj]
    {
    \hastyp{\Gamma}{m}{e}{\tau}
    }
    {
    \hastyp{\Gamma}{m}{\einj{1}{e}}{\tsum{\tau}{\tau_0}{m}}\\\\
    \hastyp{\Gamma}{m}{\einj{2}{e}}{\tsum{\tau_0}{\tau}{m}}
    } \qquad
    
    \inferrule*[lab=T-Match]
    {
    \hastyp{\Gamma}{m}{e}{\tau_0 +^m \tau_1}\\\\
    m \dashv \tau_0 \\
    m \dashv \tau_1 \\\\
    \hastyp{\Gamma,x\!:\!\tau_0}{m}{e_1}{\tau}\\
    \hastyp{\Gamma,x\!:\!\tau_1}{m}{e_2}{\tau}\\
    }
    {
      \hastyp{\Gamma}{m}{\econd{e}{e_1}{e_2}}{\tau}
    }\\ \\

    \inferrule*[lab=T-Fold]
    {
    \hastyp{\Gamma}{m}{e}{\tau[\trec{\alpha}{\tau}\setminus\alpha]}
    }
    {
    \hastyp{\Gamma}{m}{\efold{\trec{\alpha}{\tau}}{e}}{\trec{\alpha}{\tau}}
    } \qquad

    \inferrule*[lab=T-Unfold]
    {
    \hastyp{\Gamma}{m}{e}{\trec{\alpha}{\tau}}
    }
    {
    \hastyp{\Gamma}{m}{\eunfold{e}}{\tau[\trec{\alpha}{\tau}\setminus\alpha]}
    } \qquad

    \inferrule*[lab=T-Share]
    {
    P\text{ is a singleton}    \\
    m \vdash P\\\\
    \hastyp{\Gamma}{m}{e}{\tnat^P}\\
    m \vdash Q\\
    }
    {
    \hastyp{\Gamma}{m}{\eshare{\psi}{P}{Q}{e}}{\tnat^Q_\psi}
    }\\ \\
    
    \inferrule*[lab=T-Binop]
    {
    \tau = \tnat^m_\psi\\\\
    \hastyp{\Gamma}{m}{e_1}{\tau}\\\\
    \hastyp{\Gamma}{m}{e_2}{\tau}\\
    }
    {
    \hastyp{\Gamma}{m}{\ebinop{e_1}{e_2}}{\tau}
    }\qquad

    \inferrule*[lab=T-Mux]
    {
    \tau = \tnat^m_\psi\\\\
    \hastyp{\Gamma}{m}{e}{\tau}\\\\
    \hastyp{\Gamma}{m}{e_1}{\tau}\\
    \hastyp{\Gamma}{m}{e_2}{\tau}\\
    }
    {
    \hastyp{\Gamma}{m}{\emux{e}{e_1}{e_2}}{\tau}
    }\qquad

    \inferrule*[lab=T-Reveal]
    {
    m \vdash P\\
    m \vdash Q\\\\
    \hastyp{\Gamma}{m}{e}{\tnat^P_\psi}\\
    }
    {
    \hastyp{\Gamma}{m}{\ereveal{P}{Q}{e}{\psi}}{\tnat^Q}
    }\\ \\
   

   \inferrule*[lab=T-Abs]
    {
    m \vdash m'\\\\
    \hastyp{\Gamma, x\!:\!\tau_1}{m'}{e}{\tau_2}
    }
    {
    \hastyp{\Gamma}{m}{\elam{x}{e}}{\tfun{\tau_1}{\tau_2}{m'}}
    }\qquad
   
    \inferrule*[lab=T-App]
    {
    \hastyp{\Gamma}{m}{e}{\tfun{\tau_1}{\tau}{m}}\\\\
    \hastyp{\Gamma}{m}{e_1}{\tau_1}\\
    }
    {
    \hastyp{\Gamma}{m}{\eapp{e}{e_1}}{\tau}
    }\qquad

    \inferrule*[lab=T-Fix]
    {
    \hastyp{\Gamma,f\!:\! \tfun{\tau_1}{\tau_2}{m}}{m}{\elam{x}{e}}{\tfun{\tau_1}{\tau_2}{m}}
    }
    {
    \hastyp{\Gamma}{m}{\efix{f}{x}{e}}{\tfun{\tau_1}{\tau_2}{m}}
    }\\ \\

    \inferrule*[lab=T-Par]
    {
    Q = m \cap P\\
    Q \vdash \tau\\\\
    \hastyp{\Gamma}{{Q}}{e}{\tau}
    }
    {
    \hastyp{\Gamma}{m}{\epar{P}{e}}{\tau}
    } \qquad

    \inferrule*[lab=T-Sub]
    {
    \hastyp{\Gamma}{m}{e}{\tau_1}\\
    \issub{\tau_1}{\tau_2}
    }
    {
    \hastyp{\Gamma}{m}{e}{\tau_2}
    }

  \end{array}
\]
\caption{Typing rules}
\label{fig:typing}
\end{figure}

The typing judgment's rules are defined in Figure~\ref{fig:typing}.
The typing judgment has form $\hastyp{\Gamma}{m}{e}{\tau}$. It states
that under context $\Gamma$, expression $e$ has type $\tau$ when
evaluating in mode $m$. These rules reference the judgment
$m \vdash m'$ which is given in Figure~\ref{fig:aux}. It says that
that principals in $m'$ are present in mode $m$. The rules also refer
to judgment $m \vdash \tau$, which says that $\tau$ is located at
one or more principals in $m$; and $m \dashv \tau$, which says it is
located at \emph{all} principals in $m$. 

Rule~\rulelab{T-Nat} types a (cleartext) constant; it inherits the
visibility of the current mode. Rule~\rulelab{T-Read} types the
\kw{read} command; since it will produce a different value at each
location, it must be run in the visibility of a single principal ($m$
must be a singleton set). Rule~\rulelab{T-Write} is
similar. Rules~\rulelab{T-Var} and \rulelab{T-Let} are standard.

Jumping ahead, rule~\rulelab{T-Par} runs its expression $e$ in par
mode involving principals $P$. At run-time, only $m \cap P$ principals
will actually execute $e$, and so $e$ is checked in this mode, which
we name $Q$.\footnote{Wysteria requires $P \subseteq m$, which is
  always OK in \rulelab{T-Par}, too, but it's strictly more flexible.}
Likewise, the returned value must be compatible with $Q$;
i.e., it should be ``located'' at least some principals $q \in Q$.
For example, in mode $\{p\}$ we don't want to return a value located
(only) at $q$. If we did, then subsequent access by $p$ would fail at
run-time, since the value is not actually present there.
%
Compatibility $m \vdash \tau$ is given in Figure~\ref{fig:aux}. We can
think of it as related to the semantic function $\locof{m}{v}$. This
function takes $v$ and ``locates'' it at principals in $m$, filtering
out principals not present in $P$. The judgment $m \vdash \tau$
ensures that $\locof{m}{v}$ will be a no-op.

Rule~\rulelab{T-Pair} types the introduction of pairs. A pair's
components must be typeable in the current mode, but they do not
necessarily need to have the same visibility; e.g., in mode $\{p,q\}$
a pair could have one component at $p$ and the other at
$q$. Rule~\rulelab{T-Access} types the elimination of pairs. This rule
is standard.

Rules~\rulelab{T-Inj} is standard. Rule~\rulelab{T-Match} requires
that both of the to-be-accessed types are \emph{located} at all of the
principals in the current mode, per the judgment $m \dashv \tau$. This
judgment is a sort of dual to $m \vdash \tau$; it requires that all
components of $\tau$ are present to principals in $m$. We can think of
this judgment as being related to semantic function $\getat{m}{v}$: If $v$
has type $\tau$ and $m \dashv \tau$ (with the expected extensions to
typing for new value forms), then $\getat{m}{v}$ should succeed.

Intuitively, a $\tnat$ located at $\{p,q\}$ is also located at
$p$. Rule~\rulelab{T-Sub} can be used to apply this kind of
reasoning. The subtyping judgment is given in
Figure~\ref{fig:sub}. Intuitively, $\issub{\tau_1}{\tau_2}$ holds when
$\tau_1$ is compatible with the maximal mode that $\tau_2$ is
compatible with; i.e., it's OK to treat a value as being available at
fewer locations than it actually is.

The rules \rulelab{T-Fold} and \rulelab{T-Unfold} for recursive types
are standard.

Rule~\rulelab{T-Share} types encrypting a $\tnat$ (via secret
sharing). The $P$ argument indicates the principal doing the sharing,
and it must be a singleton. The $Q$ argument indicates the principals
to which to share $e$ (which must be a normal (non-share) value). This
value of $e$ must be visible in the current mode ($m \vdash P$) and
the principals $Q$ must be present as well. Finally, the value must be
a number; neither pairs nor functions can be shares.

Rule~\rulelab{T-Binop} types arithmetic computations on both shares
and normal values---both arguments must have the same type (i.e., both
shares or both normal values, with the same visibility), and match the
current mode. Note that this rule precludes adding a 
share and a normal value; you can always do this by converting the
latter to a share (the compiler can be smart about this).

Rule~\rulelab{T-Mux} types multiplexing. The semantics is to evaluate
both branches (second and third arguments), binding $x$ to the left
and right-hand sides of the sum, respectively. The $\kw{mux}$ chooses
the result to return based on the first argument's ultimate
result. These are all (compatible) shares (under protocol $\psi$), so
we can think of this as making a multiplexor circuit. The rule
restrict the results to be $\tnat$s, but this is easily generalized
via encoding (Section~\ref{sec:derived}).

Rule~\rulelab{T-Reveal} types share elimination, i.e., converting a
share of a $\tnat$ to a normal value. In essence, it forces the secure
computation (which we are thinking of as a thunk) to produce a result,
and reveals that result held by principals $P$ to the specified
principals $Q$. The rule requires that the current mode includes those
in $Q$, and those $P$ who hold the shares. (We do not allow revealing
encrypted sums, directly---we require destructing them first.)

Rule~\rulelab{T-Match} types conditionals on normal sums, i.e., the
conditional will run on each principal in mode $m$. It can return
shares or normal values, as desired. 

% Unlike the elimination rules for pairs and sums, the \rulelab{T-Var}
% does not mandate ``presence'' of its contents. So the following
% program would typecheck: $\elet{x0}{\epar{b}0}{\epar{a}{x0}}$. When
% checking the second $\kw{par}$ block, the type of $x0$ is $\tnat^{b}$,
% and this variable is occurring with a $\kw{par}$ block with only $a$
% present. But this is not prevented because nothing actually done with
% the contents of that variable.

\rulelab{T-Abs} allows the body of a function to require a strictly
smaller mode than the defining context. \mwh{Why? Can't we just wrap
  in par mode? Rethink} \rulelab{T-App} requires the caller's mode to
match the mode annotation on the function; this ensures that all
principals that must be present in the function body will indeed be
running the function. \rulelab{T-Fix} is essentially a combination of
these two, allowing $f$ to be referred to recursively in $e$.

\begin{figure}
\[\begin{array}{c}

    \inferrule*[lab=M-sub]
    {
    P \subseteq Q
    }
    {
    Q \vdash P
    } \qquad
    
    \inferrule*[lab=M-Nat]
    {
    m = n \vee
    (\psi = \cdot \Rightarrow m \vdash n)
    }
    {
    m \vdash \tnat^n_\psi
    } \qquad

    \inferrule*[lab=M-Sum]
    {
    m \vdash n\\
    m \vdash \tau_1 \\ m \vdash \tau_2
    }
    {
    m \vdash \tsum{\tau_1}{\tau_2}{n}
    } \\ \\

    \inferrule*[lab=M-Fun]
    {
    }
    {
    m \vdash \tfun{\tau_1}{\tau_2}{m}
    }\qquad

    \inferrule*[lab=M-Prod]
    {
    m \vdash \tau_1 \\ m \vdash \tau_2
    }
    {m \vdash \tprod{\tau_1}{\tau_2}}
    \qquad

    \inferrule*[lab=M-Rec]
    {m \vdash \tau}
    {m \vdash \trec{\alpha}{\tau}}
    \qquad
    
    \inferrule*[lab=m-Alpha]
    { }
    {m \vdash \alpha}
    \\ \\

    \inferrule*[lab=L-Nat]
    {  }
    {
    m \dashv \tnat^m_\psi
    } \qquad

    \inferrule*[lab=L-Sum]
    {
    m \dashv \tau_1 \\\\ m \dashv \tau_2
    }
    {
    m \dashv \tsum{\tau_1}{\tau_2}{m}
    } \qquad

    \inferrule*[lab=L-Fun]
    {
    }
    {
    m \dashv \tfun{\tau_1}{\tau_2}{m}
    }\qquad

    \inferrule*[lab=L-Prod]
    {
    m \dashv \tau_1 \\\\ m \dashv \tau_2
    }
    {m \dashv \tprod{\tau_1}{\tau_2}}
    \qquad

    \inferrule*[lab=L-Rec]
    {m \dashv \tau}
    {m \dashv \trec{\alpha}{\tau}}
    \qquad
    
    \inferrule*[lab=L-Alpha]
    { }
    {m \dashv \alpha}
    \\ \\
    
  \end{array}\]
\caption{Locatedness and Presence}
\label{fig:aux}
\end{figure}
    
\begin{figure}
\[\begin{array}{c}

    \inferrule*[lab=S-Nat]
    {
    m \vdash \tnat^n_\psi
    }
    {
    \issub{\tnat^m_\psi}{\tnat^n_\psi}
    } \qquad

    \inferrule*[lab=S-Sum]
    {
    m \vdash \tsum{\tau_1'}{\tau_2'}{n}\\\\
    \issub{\tau_1}{\tau_1'} \land \issub{\tau_2}{\tau_2'}
    }
    {
    \issub{\tsum{\tau_1}{\tau_2}{m}}{\tsum{\tau_1'}{\tau_2'}{n}}
    } \qquad
    
    \inferrule*[lab=S-Fun]
    {
    \issub{\tau_1'}{\tau_1}\\
    \issub{\tau_2}{\tau_2'}
    }
    {
    \issub{\tfun{\tau_1}{\tau_2}{m}}{\tfun{\tau_1'}{\tau_2'}{m}}
    }\qquad

    
    \inferrule*[lab=S-Pair]
    {
    \issub{\tau_1}{\tau_1'}\\
    \issub{\tau_2}{\tau_2'}
    }
    {
    \issub{\tau_1 \times \tau_2}{\tau_1' \times \tau_2'}
    }
    
\end{array}
\]
\caption{Subtyping}
\label{fig:sub}
\end{figure}

\section{Derived Forms}
\label{sec:derived}

Rather than put every useful construct in the language, we can defined
some in terms of others.

\subsection{Generalized Mux}

The $\kw{mux}$ construct is restricted to returning encrypted
$\tnat$s. We can generalize this to returning pairs of encrypted
$\tnat$s:
\begin{verbatim}
muxtopair e e1 e2 =
  let b = e in
  let (l1,r1) = e1 in
  let (l2,r2) = e2 in
  (mux b l1 l2, mux b r1 r2)
\end{verbatim}
With this encoding, we can likewise support encrypted sums, which are
encoded as pairs of encrypted nats (see below).

\mwh{It seems evident we can work this out for equal-sized lists
  too. Generalized encoding for (non-encrypted) sums?}

\subsection{Encrypted sums}

An encrypted sum is a value of the type
$\tprod{\tnat^m_\psi}{(\tprod{\tau_1}{\tau_2})}$. Here, the first
element is a boolean that represents whether the ``real'' portion of
the sum is the left side. We can introduce (encrypt) an existing sum
by doing the following:
\begin{verbatim}
let sharesum e def_lhs def_rhs =
  match e with
    x.(share 1,x,def_rhs)
  | x.(share 0,def_lhs,x)
\end{verbatim}
We can eliminate an encrypted sum by using $\kw{mux}$, which forces us
to produce a $\tnat^m_\psi$, to ensure obliviousness:
\begin{verbatim}
let muxshare e t_fun f_fun =
  let (x,(l,r)) = e in
  mux x ? t_fun l : f_fun r
\end{verbatim}

\mwh{TO DO}:
\begin{itemize}
\item I'm not sure if we literally want to encode this with functions,
  as above, or whether syntax like $\kw{match}$ which does the binding
  locally makes more sense.
\item I'm not sure about modes etc. Probably should just work out the
  derived rules for these things.
\end{itemize}
  
\subsection{Wire bundles and solo mode}

A wire bundle is just a product where each element of the product is
only visible to a single party. For simplicity, consider the smallest
one: $\tprod{\tnat^{\{p\}}_\psi}{\tnat^{\{q\}}_\psi}$. One way we
could write this as a derived type might be
$W~\{p,q\}~\tnat^\alpha_\phi$ where the interpretation is to
substitute each principal in the first argument to $W$ (a principal
set $P$) for $\alpha$ in the second argument (a type $\tau$).

To operate on a wire bundle we can use \emph{solo mode}, exemplified
in the construct $\kw{wmap}$:
\begin{verbatim}
let wmap (x:W {p,q} t) (f:t -> t0) =
  (par {p} (f (#1 x)), 
   par {q} (f (#2 x)))
\end{verbatim}
This runs the function $f$ on each element of the wire bundle, on its
respective host, and packages the results in another wire bundle. Note
that $f$ would need to be inlined at each call site because we don't
support polymorphism in the language. Moreover, $f$ might contain
occurrences of $\alpha$ which would  be substituted with
\texttt{\{p\}} in the first half of the pair, and \texttt{\{q\}} in
the second.

The above is basically Wysteria's notion of wire bundles when
constructed and operated on in $\kw{par}$ mode, and the \texttt{wmap}
above is similar to Wysteria's \texttt{waps}. Wysteria's notion of a
wire bundle in \texttt{sec} mode is basically a product of
shares, which is generated automatically by calling $\kw{share}$ on
each element of the input wire bundle. This is easily done:

\bigskip
\noindent
\texttt{wmap w }($\elam{x}{\eshare{\phi}{\alpha}{\{p,q\}}{x}}$)
\bigskip

\noindent
The \texttt{wmap} will produce a pair of shares (which . Notice the
$\alpha$ annotation on the $\kw{share}$---this will match the (solo)
mode of the principal doing the sharing, after substitution.

You can imagine generalizing this to wire bundles of arbitrary
size. You can also easily imagine converting this pair to a list, and
then supporting equivalent constructs like \texttt{wfold}.

\mwh{TO DO}
\begin{itemize}
  \item Do we want a way to run code over a wire bundle in which
    parties are not in solo mode? For example, they could run in full
    par mode (with both principals) but then accesses to the
    individual elements could be dropped to a single mode, for those
    accesses. For this, you need a map from principal $\alpha$ to
    index in the pair for $\alpha$'s value. For the accessor in the
    code, you use the map to extra from the pair. Maybe use ``self''
    and not $\alpha$. 
  \end{itemize}
  
\end{document}
